\documentclass{article}
\usepackage{graphicx} % Required for inserting images
\usepackage{amsmath}
\usepackage{dcolumn}

\title{PS11_Zilles}
\author{Andrew Zilles}
\date{April 2024}

\begin{document}


\section{Introduction}
Replication of Tax Reporting Behavior Under Audit Certainty
by Benjamin C. Ayers, Jeri K. Seidman, Erin M. Towery
Contemporary Accounting Research, 2019

Essentially, the research question is, do firms change their reporting behavior when they know for certain that they will be audited by the IRS? This question is interesting because it explores a little more in depth the role of the IRS and how companies respond to audits. It also provides a unique environment to test if audits make firms more or less risk averse.

The Ayers, Seidman, and Towery paper only looks at reported financial data. In the future I would like to expand this scope a little more and see if firms communicate differently when they are being audited vs when there is only a possibility that they might be audited. I'd like to explore the language of their 10-K filings, conference call transcripts, or other press releases to see if their tone is more limited or camouflaged in a way. (Like they're trying to "lay low" so the IRS doesn't come after them.)

\section{Literature Review}
Replication of Tax Reporting Behavior Under Audit Certainty
by Benjamin C. Ayers, Jeri K. Seidman, Erin M. Towery
Contemporary Accounting Research, 2019
And others from their paper

\section{Data}
Utilization of Compustat Fundamental Annual merged with Compustat Segments
\begin{itemize}
    \item Firm years from 2000 - 2011
    \item Merging the two databases together
    \item Creating categorical variables for the CIC program's "point system"
\end{itemize}

\section{Methods}
This is the first model I replicate from their paper. Obviously a long way to go.
\begin{equation}
\begin{aligned}
\text{CICFirm} = &\alpha + \beta_1 \cdot \text{AssetPoints} + \beta_2 \cdot \text{GrossReceiptsPoints} + \gamma_1 \cdot \text{GeoSegPoints} \\
&+ \gamma_2 \times \text{BusSegPoints} + \gamma_3 \cdot \text{ForeignSalesPoints} \\
&+ \gamma_4 \cdot \text{ForeignTaxPoints} + \varepsilon
\end{aligned}
\end{equation}


Where:
\begin{align*}
\text{CICFirm} & : \text{Indicator variable, where CICFirm equals 1 if the firm is assigned} \\
& \text{to the CIC program and 0 otherwise} \\
\alpha & : \text{Intercept} \\
\beta_1, \beta_2 & : \text{Coefficients for AssetPoints and GrossReceiptsPoints respectively} \\
\gamma_1, \gamma_2, \gamma_3, \gamma_4 & : \text{Coefficients for GeoSegPoints, BusSegPoints, ForeignSalesPoints,} \\
& \text{and ForeignTaxPoints respectively} \\
\varepsilon & : \text{Error term}
\end{align*}

The logistic regression is fitted using the binomial family with a logit link function, and the data used for modeling is from the dataset \texttt{firmyears}.

\section{Findings}
My results show that nothing is significant. But this is still very early on and I need to fine tune my models. I know I have a lot of differences with the original paper that I still need to iron out.
See Table 1 below.

% Table created by stargazer v.5.2.3 by Marek Hlavac, Social Policy Institute. E-mail: marek.hlavac at gmail.com
% Date and time: Tue, Apr 23, 2024 - 3:26:45 PM
% Requires LaTeX packages: dcolumn 
\begin{table}[!htbp] \centering 
  \caption{Logistic Regression Results} 
  \label{} 
\begin{tabular}{@{\extracolsep{5pt}}lD{.}{.}{-3} }
\\[-1.8ex]\hline 
\hline \\[-1.8ex] 
 & \multicolumn{1}{c}{\textit{Dependent variable:}} \\ 
\cline{2-2} 
\\[-1.8ex] & \multicolumn{1}{c}{CICFirm} \\ 
\hline \\[-1.8ex] 
 AssetPoints & 41.092 \\ 
  & (449.795) \\ 
  & \\ 
 GrossReceiptsPoints & 41.211 \\ 
  & (472.529) \\ 
  & \\ 
 GeoSegPoints & 41.203 \\ 
  & (462.298) \\ 
  & \\ 
 BusSegPoints & 41.104 \\ 
  & (452.480) \\ 
  & \\ 
 ForeignSalesPoints & 39.925 \\ 
  & (753.931) \\ 
  & \\ 
 ForeignTaxPoints & 41.177 \\ 
  & (758.447) \\ 
  & \\ 
 Constant & -471.954 \\ 
  & (4,792.371) \\ 
  & \\ 
\hline \\[-1.8ex] 
Observations & \multicolumn{1}{c}{47,873} \\ 
Log Likelihood & \multicolumn{1}{c}{-0.00001} \\ 
Akaike Inf. Crit. & \multicolumn{1}{c}{14.000} \\ 
\hline 
\hline \\[-1.8ex] 
\textit{Note:}  & \multicolumn{1}{r}{$^{*}$p$<$0.1; $^{**}$p$<$0.05; $^{***}$p$<$0.01} \\ 
\end{tabular} 
\end{table}

\section{Conclusion}
I conclude that there's more work to do

\end{document}
